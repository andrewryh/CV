%%%%%%%%%%%%%%%%%%%%%%%%%%%%%%%%%%%%%%%%%
% Plain Cover Letter
% LaTeX Template
%
% This template has been downloaded from:
% http://www.latextemplates.com
%
% Original author:
% Rensselaer Polytechnic Institute (http://www.rpi.edu/dept/arc/training/latex/resumes/)
% It was modified by Katy Huff
%
%%%%%%%%%%%%%%%%%%%%%%%%%%%%%%%%%%%%%%%%%

%----------------------------------------------------------------------------------------
%    PACKAGES AND OTHER DOCUMENT CONFIGURATIONS
%----------------------------------------------------------------------------------------

\documentclass[11pt]{letter} % Default font size of the document, change to 10pt to fit more text
\usepackage{graphicx}
%\usepackage{newcent} % Default font is the New Century Schoolbook PostScript font
%\usepackage{helvet} % Uncomment this (while commenting the above line) to use the Helvetica font

% Margins
\usepackage[left=1.25in,right=1.25in,top=1in,bottom=1in]{geometry}
%\let\raggedleft\raggedright % Pushes the date (at the top) to the left, comment this line to have the date on the right

\usepackage{eso-pic,graphicx}


%--------------------------------------------------------------------------------------
%--------INPUT DATA
%--------------------------------------------------------------------------------------
\usepackage{xspace}
\newcommand{\StudentFirstName}{Andrei\xspace}
\newcommand{\StudentLastName}{Rykhlevskii\xspace}
%\newcommand{\RecipientName}{Dr. Germina Ilas\xspace}
\newcommand{\RecipientName}{Recruiting Manager\xspace}
\newcommand{\RecipientAddress}{Kairos Power, LLC}
%\newcommand{\<++>}{<++>\xspace}

\begin{document}
\AddToShipoutPictureBG*{\includegraphics[width=\paperwidth,height=\paperheight]{background.pdf}}


%----------------------------------------------------------------------------------------
%    ADDRESSEE SECTION
%----------------------------------------------------------------------------------------

\begin{letter}{\RecipientName\\
        \RecipientAddress\xspace}

%-------------------------------------------------------------------------------
%    YOUR NAME & ADDRESS SECTION
%-------------------------------------------------------------------------------
\address{Andrei Rykhlevskii\\
andreir2@illinois.edu\\
226 Talbot Laboratory\\
104 Wright Street\\
Urbana, IL 61801}

%-------------------------------------------------------------------------------
%    LETTER CONTENT SECTION
%-------------------------------------------------------------------------------

\opening{To \RecipientName,}

I am responding to the announcement at www.kairospower.com for a Sr Nuclear 
Engineer -- Reactor Analyst position. I am a Ph.D. Candidate in the Nuclear, 
Plasma, \& Radiological Engineering Department at the University of Illinois 
at Urbana-Champaign. My current research and dissertation focus is on 
developing computational tools for neutronics and fuel cycle modeling of 
molten salt nuclear reactors (MSRs).  I received my M.Sc. in Nuclear 
Engineering with a focus on Reactor Physics from the UIUC in May of 2018 and 
expect to complete my Ph.D. in Nuclear Engineering with Computational Science 
and Engineering (CSE) Concentration in Summer of 2020.

I earned a bachelor’s degree in 2010 from Bauman Moscow State Technical 
University, concentrating on reactor physics and nuclear fuel cycle. After 
graduation, I worked as a nuclear reactor analyst in ROSATOM for a total of 3 
years. My research has primarily been in neutron transport, mesh generation, 
and safety analysis for light-water reactors (VVER). I also was involved in 
validation and Verification effort for $S_N$ transport software KATRIN (part 
of CNCSN 2009).

In my Ph.D. work, I am developing tool\footnote{SaltProc is open-source 
software and available on Github: https://github.com/arfc/saltproc/} to 
perform burn-up simulations for liquid-fueled MSR with taking into account 
on-line reprocessing and refueling. This work includes uncertainty 
quantification for the depleted fuel composition due to uncertainties in 
neutron transport problem solution and  nuclear data. Additionally, I worked 
on the verification, validation, and application of Moltres\footnote{Moltres 
is open-source software and available on Github: 
https://github.com/arfc/moltres/}, a MOOSE application designed for 
multi-physics simulation of liquid-fueled MSRs. Moltres solves arbitrary-group 
neutron diffusion, temperature, and precursor governing  equations in three 
dimensions. We demonstrated Moltres capabilities for the Molten Salt Reactor 
Experiment (MSRE) and European Molten Salt Fast Reactor (MSFR).

My work has resulted in 2 journal publications (+2 submitted) and 3 full 
conference papers. My publication record in computational nuclear systems 
analysis and reactor modeling, as well as 3 years of experience in ROSATOM, 
have helped me gain a skill set that will readily align with the goals of the 
Kairos Power. Additionally, my experience in confirmatory analysis assessments 
for the Russian Federal Service for Environmental, Technological, and Nuclear 
Supervision, as well as expertise in using Serpent, SCALE, MCNP, OpenMC, and 
MOOSE promise confidence and ability that will be necessary to support the 
Kairos Power mission. 

I am especially interested in the possibility of creating a detailed models of 
the KP-FHR core using advanced neutronics and multi-physics software. Core 
design optimization and safety analysis of the KP-FHR is an exciting task I 
would be thrilled to tackle. I am also excited to learn and contribute 
to new codes and methods. Finally, my Ph.D. training includes a series of 
advanced graduate numerical methods and parallel algorithms courses which can 
be useful for developing advanced scientific software.

Irrespective of the status of my application, I would be happy to share my 
most recent research results and learn more about Kairos Power. Thank you for 
your consideration.

\closing{Sincere regards,\\
{Andrei Rykhlevskii\\
Ph.D. Candidate\\
Nuclear, Plasma, \& Radiological Eng.\\
U. Illinois at Urbana-Champaign}
}

%-------------------------------------------------------------------------------

\end{letter}

\end{document}



