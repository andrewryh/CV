%%%%%%%%%%%%%%%%%%%%%%%%%%%%%%%%%%%%%%%%%
% Plain Cover Letter
% LaTeX Template
%
% This template has been downloaded from:
% http://www.latextemplates.com
%
% Original author:
% Rensselaer Polytechnic Institute (http://www.rpi.edu/dept/arc/training/latex/resumes/)
% It was modified by Katy Huff
%
%%%%%%%%%%%%%%%%%%%%%%%%%%%%%%%%%%%%%%%%%

%----------------------------------------------------------------------------------------
%    PACKAGES AND OTHER DOCUMENT CONFIGURATIONS
%----------------------------------------------------------------------------------------

\documentclass[11pt]{letter} % Default font size of the document, change to 10pt to fit more text
\usepackage{graphicx}
%\usepackage{newcent} % Default font is the New Century Schoolbook PostScript font
%\usepackage{helvet} % Uncomment this (while commenting the above line) to use the Helvetica font

% Margins
\usepackage[left=1.25in,right=1.25in,top=1in,bottom=1in]{geometry}
%\let\raggedleft\raggedright % Pushes the date (at the top) to the left, comment this line to have the date on the right

\usepackage{eso-pic,graphicx}


%--------------------------------------------------------------------------------------
%--------INPUT DATA
%--------------------------------------------------------------------------------------
\usepackage{xspace}
\newcommand{\StudentFirstName}{Andrei\xspace}
\newcommand{\StudentLastName}{Rykhlevskii\xspace}
%\newcommand{\RecipientName}{Dr. David F. Wang\xspace}
\newcommand{\RecipientName}{Recruiting Manager\xspace}
\newcommand{\RecipientAddress}{Nuclear Science and Engineering Division \\
Argonne National Laboratory}
%\newcommand{\<++>}{<++>\xspace}

\begin{document}
\AddToShipoutPictureBG*{\includegraphics[width=\paperwidth,height=\paperheight]{background.pdf}}


%----------------------------------------------------------------------------------------
%    ADDRESSEE SECTION
%----------------------------------------------------------------------------------------

\begin{letter}{\RecipientName\\
        \RecipientAddress\xspace}

%-------------------------------------------------------------------------------
%    YOUR NAME & ADDRESS SECTION
%-------------------------------------------------------------------------------
\address{Andrei Rykhlevskii\\
andreir2@illinois.edu\\
226 Talbot Laboratory\\
104 Wright Street\\
Urbana, IL 61801}

%-------------------------------------------------------------------------------
%    LETTER CONTENT SECTION
%-------------------------------------------------------------------------------

\opening{To \RecipientName,}

I am responding to the announcement at www.anl.gov for a Postdoctoral 
appointee 
(406750) position in the Nuclear Science and Engineering Division. I am a 
Ph.D. Candidate in the Nuclear, Plasma, \& Radiological Engineering Department 
at the University of Illinois at Urbana-Champaign. My current research and 
dissertation focus is on developing computational tools for neutronics and 
fuel cycle modeling of molten salt nuclear reactors.  I received my M.Sc. in 
Nuclear Engineering with a focus on Reactor Physics from the UIUC in May of 
2018 and expect to complete my Ph.D. in Nuclear Engineering with Computational 
Science and Engineering (CSE) Concentration in Summer of 2020.

I earned a bachelor’s degree in 2010 from Bauman Moscow State Technical 
University, with a focus on Reactor Physics and Nuclear Fuel Cycle. After 
graduation, I worked as a nuclear reactor analyst in ROSATOM for a total of 3 
years. My research has primarily been in the development of neutronics models 
and codes for light-water reactors (VVER).

At ROSATOM, I worked on the verification, validation, and application of 
KATRIN (part of CNCSN 2009),  parallel $S_N$ transport code for fluence 
calculations. KATRIN’s geometry is combinatorial, and the partly-structured 
mesh is manually generated to fit it. I developed several scripts and codes 
for the generation of such meshes. This required me to gain intuition about 
local mesh refinement, as well as  made me better at parametrizing complex 
shapes. Additionally, I performed a full-VVER vessel solution of Bateman 
equations, using KATRIN and DORT/TORT.

My strong publication record in computational nuclear systems analysis and 
reactor modeling, as well as 3 years of experience in ROSATOM, have helped me 
gain a skill set that will readily align with the goals of the Nuclear Science 
and Engineering Division. Additionally, my experience in the confirmatory 
analysis assessments for the Russian Federal Service for Environmental, 
Technological, and Nuclear Supervision, as well as expertise in using SCALE, 
MCNP, Serpent, OpenMC promise confidence and ability that will be necessary to 
support the Argonne National Laboratory mission. I also have experience of 
using CYCLUS, which was developed by my Ph.D. advisor, Kathryn D. Huff, and 
our research group continuously developing and improving this Fuel Cycle 
Simulator.

Irrespective of the status of my application, I would be happy to share my 
most recent research results and learn more about the Nuclear Science and 
Engineering Division at ANL. Thank you for your consideration.

\closing{Sincere regards,\\
{Andrei Rykhlevskii\\
Ph.D. Candidate\\
Nuclear, Plasma, \& Radiological Eng.\\
U. Illinois at Urbana-Champaign}
}

%-------------------------------------------------------------------------------

\end{letter}

\end{document}



