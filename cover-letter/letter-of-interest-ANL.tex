%%%%%%%%%%%%%%%%%%%%%%%%%%%%%%%%%%%%%%%%%
% Plain Cover Letter
% LaTeX Template
%
% This template has been downloaded from:
% http://www.latextemplates.com
%
% Original author:
% Rensselaer Polytechnic Institute (http://www.rpi.edu/dept/arc/training/latex/resumes/)
% It was modified by Katy Huff
%
%%%%%%%%%%%%%%%%%%%%%%%%%%%%%%%%%%%%%%%%%

%----------------------------------------------------------------------------------------
%    PACKAGES AND OTHER DOCUMENT CONFIGURATIONS
%----------------------------------------------------------------------------------------

\documentclass[11pt]{letter} % Default font size of the document, change to 10pt to fit more text
\usepackage{graphicx}
%\usepackage{newcent} % Default font is the New Century Schoolbook PostScript font
%\usepackage{helvet} % Uncomment this (while commenting the above line) to use the Helvetica font

% Margins
\usepackage[left=1.25in,right=1.25in,top=1in,bottom=1in]{geometry}
%\let\raggedleft\raggedright % Pushes the date (at the top) to the left, comment this line to have the date on the right

\usepackage{eso-pic,graphicx}


%--------------------------------------------------------------------------------------
%--------INPUT DATA
%--------------------------------------------------------------------------------------
\usepackage{xspace}
\newcommand{\StudentFirstName}{Andrei\xspace}
\newcommand{\StudentLastName}{Rykhlevskii\xspace}
%\newcommand{\RecipientName}{Dr. David F. Wang\xspace}
\newcommand{\RecipientName}{Recruiting Manager\xspace}
\newcommand{\RecipientAddress}{Nuclear Science and Engineering Division \\
Argonne National Laboratory}
%\newcommand{\<++>}{<++>\xspace}

\begin{document}
\AddToShipoutPictureBG*{\includegraphics[width=\paperwidth,height=\paperheight]{background.pdf}}


%----------------------------------------------------------------------------------------
%    ADDRESSEE SECTION
%----------------------------------------------------------------------------------------

\begin{letter}{\RecipientName\\
        \RecipientAddress\xspace}

%-------------------------------------------------------------------------------
%    YOUR NAME & ADDRESS SECTION
%-------------------------------------------------------------------------------
\address{Andrei Rykhlevskii\\
andreir2@illinois.edu\\
226 Talbot Laboratory\\
104 Wright Street\\
Urbana, IL 61801}

%-------------------------------------------------------------------------------
%    LETTER CONTENT SECTION
%-------------------------------------------------------------------------------

\opening{To \RecipientName,}

In 2010, I graduated from the Bauman Moscow State Technical University (Bauman 
MSTU, Moscow, Russia) with a Specialist’s (MS-equivalent) degree in Nuclear 
Engineering, and worked as a nuclear reactor analyst for ROSATOM almost 3 
years. There, my research focused on neutron transport and mesh generation in 
the neutron transport code KATRIN, with a focus on verification and 
validation. In August 2016, I joined Professor Huff’s group at the University 
of Illinois at Urbana-Champaign (UIUC). My current research focuses on 
developing computational tools for online reprocessing simulation and safety 
analysis for advanced molten salt nuclear reactors (MSR). In my MS thesis 
(completed in May 2018), I illuminated the online reprocessing problem in the 
MSBR through computational analysis using SaltProc, a python package I 
developed to simulate online fuel salt reprocessing\footnote{SaltProc is 
open-source software and available on Github: 
https://github.com/arfc/saltproc/}. This tool models online salt treatment and 
automates iterative depletion calculations via coupling with the SERPENT2 
Monte Carlo code. My dissertation focuses on developing computational tools 
for neutronics and fuel cycle modeling of molten salt nuclear reactors. I 
expect to complete my Ph.D. in Nuclear Engineering with Computational Science 
and Engineering (CSE) Concentration in the Summer of 2020.

Additionally, I worked on the verification, validation, and application of 
Moltres\footnote{Moltres is open-source software and available on Github: 
https://github.com/arfc/moltres/}, a MOOSE application designed for 
multi-physics simulation of molten salt reactors. Moltres solves 
arbitrary-group neutron diffusion, temperature, and precursor governing 
equations in three dimensions. Moltres capabilities have been demonstrated for 
the Molten Salt Reactor Experiment (MSRE) and European Molten Salt Fast 
Reactor (MSFR).

Over summer 2018 I had the privilege to be an intern at Oak Ridge National 
Laboratory, in the Reactor and Nuclear Systems Division, Reactor Physics 
group. There, I developed full-core and simplified unit cell models of four 
prospective Fast Molten Salt Reactors using various transport software. Those 
models have been used for lifetime-long depletion simulation with realistic 
truly continuous online reprocessing. Based on these simulations, Fuel Cycle 
Evaluation Metrics have been calculated and for comparison among competitive 
fuel cycle technologies. This work culminated in conference submissions to 
M\&C2019 and GLOBAL2019. Potentially, this data may be used to discuss U.S. 
future transition scenarios from decommissioned LWRs to Generation IV MSRs. 
This short-term internship inspired me to seek an opportunity for long-term 
employment in the U.S. National Laboratories system.

My work has resulted in 2 journal publications and 3 refereed conference 
papers. My publication record in computational nuclear systems analysis and 
reactor modeling, as well as 3 years of experience in ROSATOM, have helped me 
gain skills that will readily align with the mission of the Nuclear Science 
and Engineering Division at ANL. Additionally, my experience in coupled 
multi-physics simulation, as well as expertise in using Serpent, SCALE, 
OpenMC, and MCNP, promise confidence and ability that will be necessary to 
support the ANL mission. 

I am especially interested in the possibility of creating a detailed models of 
the Versatile Test Reactor (VTR) design using advanced neutronics and 
multi-physics software. Core design optimization and safety analysis of a fast 
spectrum sodium-cooled testing facility of this kind is an exciting task I 
would be thrilled to tackle. I am also excited to learn new software such as 
DIF3D, VIM, MC2, SHARP, SAS4A/SASSYS. Finally, my Ph.D. training includes a 
series of advanced graduate numerical methods and parallel algorithms courses 
which can be useful for developing or coupling advanced scientific software.

Irrespective of the status of my application, I would be happy to share my 
most recent research results and learn more about the Nuclear Science and 
Engineering Division at ANL. Thank you for your consideration.


\closing{Sincere regards,\\
{Andrei Rykhlevskii\\
Ph.D. Candidate\\
Nuclear, Plasma, \& Radiological Eng.\\
U. Illinois at Urbana-Champaign}
}

%-------------------------------------------------------------------------------

\end{letter}

\end{document}



