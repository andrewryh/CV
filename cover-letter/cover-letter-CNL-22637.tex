%%%%%%%%%%%%%%%%%%%%%%%%%%%%%%%%%%%%%%%%%
% Plain Cover Letter
% LaTeX Template
%
% This template has been downloaded from:
% http://www.latextemplates.com
%
% Original author:
% Rensselaer Polytechnic Institute (http://www.rpi.edu/dept/arc/training/latex/resumes/)
% It was modified by Katy Huff
%
%%%%%%%%%%%%%%%%%%%%%%%%%%%%%%%%%%%%%%%%%

%----------------------------------------------------------------------------------------
%    PACKAGES AND OTHER DOCUMENT CONFIGURATIONS
%----------------------------------------------------------------------------------------

\documentclass[11pt]{letter} % Default font size of the document, change to 10pt to fit more text
\usepackage{graphicx}
%\usepackage{newcent} % Default font is the New Century Schoolbook PostScript font
%\usepackage{helvet} % Uncomment this (while commenting the above line) to use the Helvetica font

% Margins
\usepackage[left=1.25in,right=1.25in,top=1in,bottom=1in]{geometry}
%\let\raggedleft\raggedright % Pushes the date (at the top) to the left, comment this line to have the date on the right

\usepackage{eso-pic,graphicx}


%--------------------------------------------------------------------------------------
%--------INPUT DATA
%--------------------------------------------------------------------------------------
\usepackage{xspace}
\newcommand{\StudentFirstName}{Andrei\xspace}
\newcommand{\StudentLastName}{Rykhlevskii\xspace}
\newcommand{\RecipientName}{Dr. David F. Wang\xspace}
%\newcommand{\RecipientName}{Recruiting Manager\xspace}
\newcommand{\RecipientAddress}{Systems and Safety Analysis Branch \\
Canadian Nuclear Laboratories}
%\newcommand{\<++>}{<++>\xspace}

\begin{document}
\AddToShipoutPictureBG*{\includegraphics[width=\paperwidth,height=\paperheight]{background.pdf}}


%----------------------------------------------------------------------------------------
%    ADDRESSEE SECTION
%----------------------------------------------------------------------------------------

\begin{letter}{\RecipientName\\
        \RecipientAddress\xspace}

%-------------------------------------------------------------------------------
%    YOUR NAME & ADDRESS SECTION
%-------------------------------------------------------------------------------
\address{Andrei Rykhlevskii\\
andreir2@illinois.edu\\
226 Talbot Laboratory\\
104 Wright Street\\
Urbana, IL 61801}

%-------------------------------------------------------------------------------
%    LETTER CONTENT SECTION
%-------------------------------------------------------------------------------

\opening{Dear \RecipientName,}

I am responding to the announcement at www.cnl.ca for an Research Scientist  
(22637) position in the Systems \& Safety Analysis Branch. I am a Ph.D. 
Candidate in the Nuclear, Plasma, \& Radiological Engineering Department at 
the University of Illinois at Urbana-Champaign. My current research and 
dissertation focus is on developing computational tools for neutronics and 
fuel cycle modelling of molten salt nuclear reactors.  I received my M.Sc. in 
Nuclear Engineering with a focus on Reactor Physics from the UIUC in May of 
2018 and expect to complete my Ph.D. in Nuclear Engineering with Computational 
Science and Engineering Concentration in Summer of 2020.

I earned a bachelor’s degree in 2010 from Bauman Moscow State Technical 
University, with a focus on Reactor Physics and Nuclear Fuel Cycle. After 
graduation, I worked as a nuclear reactor analyst in ROSATOM for a total of 3 
years. My research has primarily been in design and safety analysis for 
light-water reactors (VVER) using various reactor physics and radiation 
transport software.

One of the projects was on the topic of a full-VVER vessel solution of Bateman 
equations, using KATRIN and WIMS to determine activation of structural 
material. Additionally, I performed safety analysis and wrote 
a Decommissioning Chapter of Preliminary Safety Analysis Report (PSAR) for 
Belene NPP.

My strong publication record in computational nuclear systems analysis and 
reactor modelling, as well as 3 years of experience in ROSATOM, have helped me 
gain a skill set that will readily align with the goals of the Systems \& 
Safety Analysis Branch. Additionally, my experience in confirmatory analysis 
assessments for Russian Federal Service for Environmental, Technological, and 
Nuclear Supervision, as well as expertise in using Serpent, SCALE, OpenMC, 
MCNP promise confidence and ability that will be necessary to support the 
CNL's strategic initiatives. 

Irrespective of the status of my application, I would be happy to share my 
most recent research results and learn more about the Systems \& Safety 
Analysis Branch at CNL. Thank you for your consideration.

\closing{Sincere regards,\\
{Andrei Rykhlevskii\\
Ph.D. Candidate\\
Nuclear, Plasma, \& Radiological Eng.\\
U. Illinois at Urbana-Champaign}
}

%-------------------------------------------------------------------------------

\end{letter}

\end{document}



